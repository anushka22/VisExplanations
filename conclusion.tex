\section{Conclusion}

Small multiple displays are a powerful mechanism to analyze subsets of a visualized data relationship conditioned on another variable(s). Multidimensional datasets offer a challenge due to the combinatorics of the choice of variables for such a partitioning of the data. In this paper, we described a method for automatically ranking the small multiple displays created by the partitioning variables in a data set. We described a set of goodness criteria for small multiple displays that favors fewer partitions, have visually rich patterns that are well-supported by data observations and are different from the patterns seen in the unpartitioned view of the same data.

Here, we focus on scatterplots, as the primary data view, and scagnostics, as measures of visual patterns, to illustrate our method of evaluating small multiple displays. Our method can incorporate a wide range of existing quality measures for different visualization types, making our approach very general. Our use of a randomized permutation test allows our method to detect and discount non-informative or spurious patterns in small multiple displays.

The basis of our approach---the combination of cognostics and non-parametric tests---is very general and there is much more work to be done exploring this area. Such approaches will provide visualization users with rich tools that help them explore their data faster and more accurately.

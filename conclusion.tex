\section{Conclusion}

Small multiple displays are a powerful mechanism to analyze a visual relationship conditioned on other variables. Multidimensional datasets offer a challenge due to the combinatorics in the choice of partitioning variables. In this paper, made a first step in addressing this problem by describing a method for automatically ranking the small multiple displays created by the partitioning variables in a data set.
Our use of a randomized permutation test allows our method to detect and discount non-informative or spurious patterns in small multiple displays.
We also described a set of goodness criteria for small multiple displays that favors fewer partitions, visually rich patterns that are well-supported by data observations and are different from the patterns seen in the unpartitioned view of the same data.

The basis of our approach---the combination of cognostics and non-parametric tests---is very general and, as we have outlined, there is much more work to be done exploring this area. We focused on scatterplots, as the primary data view, and scagnostics, as measures of visual patterns, to illustrate our method of evaluating small multiple displays. But, our method can incorporate a wide range of quality measures allowing it to be used on different visualization type and to address different analytic goals. We believe that the development of new cognostics and the application of them to visual analytics using non-parameteric approaches will provide analysts with a new generation of tools that will help them explore their data faster and more accurately.


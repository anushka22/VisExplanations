\section{Related Work}
\label{sec:related}
Previous work in this area falls into three categories. First, we discuss previous work on generating small multiple displays. We then discuss related work on metrics for evaluatingmeasure the quality or usefulness of a variety of visualization types. Finally, we describe other applications of non-parametric statistics to visualization.

\subsection{Small Multiple Displays}
Small multiple displays are tables of similar visualizations, where each cell visualizes a subset of the data. Such displays support visual comparisons across subsets of the data; allowing viewers to make visual inferences about the conditional impact of the partitioning variable(s). Small multiples have several centuries of usage dating back to the work of economist W. S. Jevons in the 19th century~\cite{Kelley1973} who transformed tables of time series data into graphical explanations. 

Many popular visual analysis tools can generate small multiple displays, such as the Trellis package in S-Plus~\cite{Becker1996-manual}, the ggplot2 library in the R language~\cite{Wickham2006}, and the Polaris system~\cite{Stolte2002} (now Tableau). These systems allow users to rapidly generate small multiple displays to explore data. However, the tools require users to manually select partitioning variables. Mackinlay's APT system~\cite{mackinlay1986} and Tableau's Show Me system~\cite{mackinlay2007} suggests heuristics for effectively laying out small multiples based on the data types and functional dependencies in a data set. Users must provide the partitioning variables as input to these algorithms. 

MacEachren et. al~\cite{Maceachren2003} use conditional entropy to identify pairs of variables in a high-dimensional dataset that are likely to display interesting relationships. These variables are displayed in matrix of similar or different view types. Here too, the choice of a conditioning, non-displayed variable to filter or focus these views is at the user's discretion. Trelliscope~\cite{Hafen2013} provides an R package to efficiently generate and organize the large number of panels that result from using a trellis display on complex data. They leverage user-specified cognostics~\cite{Tukey1982}, ``computer guiding diagnostics" or measures of the usefulness of a data view, to sample, filter and sort panels that are potentially ``interesting" to a user. 

%Small multiple displays have been effectively used in a wide range of analytic domains including geography~\cite{Guo2006, Maceachren2003}, medicine~\cite{Lunzer2010, Sarni2005}, and analysis of designed experiments~\cite{Fuentes2011}.

Small multiples are also employed as a visual layout metaphor in user interfaces for exploring the input space of visual parameters as in Design Galleries \cite{marks1997}. The alternating use of small multiples together with a large single view is used as an interaction device for data exploration \cite{van2013}. However, our work focuses on small multiples that represent partitions from faceting~\cite{Wilkinson2005GG} a set of data.

\subsection{Visualization Quality Measures}
While our work focuses on automatic selection of small multiple displays, there is substantial previous work focused on the development of quality measures for a wide range of other visualization types. 

Scagnostics---quality measures for scatterplots---were first proposed by Tukey and Tukey~\cite{Tukey1982, Tukey1985}. More computationally efficient scagnostics were proposed by Wilkinson et al.~\cite{Wilkinson2005, Wilkinson2008}. Other measures have focused on particular scatterplot tasks, such as cluster separation~\cite{Sedlmair2012, Tatu2009}, class consistency and separation~\cite{Sips2009, Schafer2013}, or statistical properties~\cite{Kandel2012, Seo2005, Piringer2008}.
Quality measures have also been developed for other plot types. Many authors have suggested measures for parallel coordinate plots~\cite{Ankerst1998, Dasgupta2010, Johansson2009, Yang2003}. Albuquerque et al.~\cite{Albuquerque2010} offer quality measures for radial visualizations, pixel-oriented displays, and table lenses. Schneidewind et al.\ propose Pixnostics~\cite{Schneidewind2006} a quality measure based directly on the pixel values of a visualization. Some measures~\cite{Bertini2006, Cui2006, Yang2003} focus on the level of abstraction, including aggregation, clustering, and sampling in these chart types. For more information on quality measures for visualization, consult the survey by Bertini et al.~\cite{Bertini2011}. 

Some visual analytic systems leverage these measures to recommend visualizations to their users. For example, ScagExplorer~\cite{Dang2014} applies scagnostics to cluster and filter through large collections of bivariate relationships automatically.
EvoGraphDice~\cite{Boukhelifa2013} uses evolutionary algorithms and a scagnostics-based fitness function to select interesting linear and non-linear 2D projections.
AutoVis~\cite{Wills2010} applies scagnostics to make decisions about what to visualize to provide users a first glance at their data, highlighting patterns that statisticians notice when investigating datasets.

\subsection{Non-parametric approaches in visualization}
We use a non-parametric permutation test to rank small multiple displays. Other researchers have explored other uses for non-parametric statistical methods in visualizations.

Graphical inference~\cite{Buja2009, Wickham2013, Majumder2013} asks viewers to judge whether a visualization of the actual dataset is visually distinguishable from random bootstrapped samples. The result is a non-parametric significance test of a visual pattern. Menjoge~\cite{Menjoge2010} applies that idea and uses bootstrapping to generate a 95\% visual confidence interval that can correctly communicate the sampling variability in a visual pattern. 


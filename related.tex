\section{Related Work}
In this section, we first summarize previous work in three related areas: visual inference methods, quality metrics for data visualizations and work related to Scagnostics.

\subsection{Visual Inference Methods}
Bridging the gap between exploratory and confirmatory statistics is work that investigates statistical significance testing in the hypothesis testing of visual findings~\cite{Wickham2013,Majumder2013}. Human subjects are asked whether the observed dataset looks anything like random bootstrapped samples in lineup or Rorschach protocols to enable simulation based statistical inference of visual patterns~\cite{Buja2009}.

"New Procedures for Visualizing Data and Diagnosing Regression Models" by R Menjoge. This is a thesis. Chapter 2 has a very interesting approach using bootstrapping to get a 95\% visual confidence interval for a single plot. It's like the effect size version of Buja's null hypothesis approach. I can't find a published paper other than the thesis.

\subsection{Data Visualization Quality Metrics}
A large number of quality metrics are used in methods for high-dimensional data analysis \cite{Bertini2011}.  Many metrics for scatterplots determine projections of the data to be displayed, often for particular tasks: cluster separation~\cite{Sedlmair2012,Tatu2009}, class consistency and separation~\cite{Sips2009,Schafer2013}, interesting visual shapes~\cite{Wilkinson2005} or statistical properties~\cite{Kandel2012,Seo2005}. Metrics for parallel coordinate plots~\cite{Ankerst1998, Dasgupta2010, Johansson2009, Yang2003} also focus on the ordering of dimensions. Some metrics~\cite{Bertini2006, Cui2006} focus on the level of abstraction, including aggregration and sampling, in these chart types. Others~\cite{Albuquerque2010, Ankerst1998, Schneidewind2006, Yang2003} offer metrics for radial displays, pixel maps, table lens and other visualization types. 

Tableau's Show Me \cite{mackinlay2007} suggests appropriate chart types based on a set of two or more user-selected data fields. When the user selects a chart type, the system automatically creates a view for that chart type. However, for each chart type there are multiple possible effective views based on different visual mappings and dimension orderings of the selected fields. Our work extends automatic data visualization generation to support small multiples alternatives by determining appropriate visual encodings, level of abstraction and ordering of data fields. We contribute scoring criteria to rank data views considering data properties and visual layout.


Rank-by-feature~\cite{Seo2005} (rank all 2D projections). a follow up~\cite{Piringer2008} that claims they do rank-by-feature on subsets and compares subsets. I thought this would be closely related to our work, but I can't figure out what the paper is doing.

Evolutionary selection~\cite{Boukhelifa2013} of linear and nonlinear 2D projections.

\subsection{Scagnostics}
Graph-theoretic scagnostics~\cite{Wilkinson2005}, Scagnostic distributions~\cite{Wilkinson2008}, ScagExplorer~\cite{Dang2014}

Experimental evaluation~\cite{Lehmann2015} of scagnostics and other metrics, showing that scagnostics best matches human rankings.

AutoVis~\cite{Wills2010}


\section{Related Work}
\label{sec:related}
Previous work in this area falls into three categories. First, we discuss previous work on generating small multiple displays. We then discuss related work on metrics for measuring the quality or usefulness of a variety of visualization types. Finally, we describe other applications of non-parameteric statistics to visualization.

\subsection{Small Multiple Displays}
Small multiple displays are tables of similar visualizations, where each cell visualizations a subset of the data. Such displays support visual comparisons across subsets of the data; allowing viewers to make visual inferences about the conditional impact of the partitioning variable(s).

(Need some history of small multiples references here.)

Many popular visual analysis tools can generate small multiple displays, such as the Trellis package in S-Plus~\cite{}, the ggplot2 library in the R language~\cite{Wickham2006}, and the Polaris system~\cite{Stolte2002} (now Tableau). These systems allow users to rapidly generate small multiple visualizations to explore data. However, the tools require users to manually select partitioning variables. Mackinlay's APT system~\cite{} and Tableau's Show Me system~\cite{mackinlay2007} suggests heuristics for effectively laying out small multiples based on the types and functional dependencies in a data set. Users must provide the partitioning variables as input to these algorithms. 

%Small multiple displays have been effectively used in a wide range of analytic domains including geography~\cite{Guo2006, Maceachren2003}, medicine~\cite{Lunzer2010, Sarni2005}, and analysis of designed experiments~\cite{Fuentes2011}.

(Need to make sure we've cited any small multiple display work from the last 3 or 4 years in InfoVis.)

%Faceting is a slicing operator~\cite{Wilkinson2005GG, Munzner2014} to split up a dataset into subsets that are examined together. Often these subsets are determined by discrete values of a categorical dimension or bins of a quantitative dimension.

\subsection{Visualization Quality Metrics}
While our work focuses on automatic selection of small multiple displays, there is substantial previous work focused on the development of quality metrics for a wide range of other visualization types. 

Scagnostics---quality metrics for scatterplots---were first proposed by Tukey and Tukey~\cite{Tukey1982, Tukey1985}. More computationally efficient scagnostics were proposed by Wilkinson et al.~\cite{Wilkinson2005, Wilkinson2008}. Other metrics have focused on particular scatterplot tasks, such as cluster separation~\cite{Sedlmair2012, Tatu2009}, class consistency and separation~\cite{Sips2009, Schafer2013}, or statistical properties~\cite{Kandel2012, Seo2005, Piringer2008}.
Quality metrics have also been developed for other plot types. Many authors have suggested metrics for parallel coordinate plots~\cite{Ankerst1998, Dasgupta2010, Johansson2009, Yang2003}. Albuquerque et al.~\cite{Albuquerque2010} offer quality metrics for radial visualizations, pixel-oriented displays, and table lenses. Schneidewind et al.\ propose Pixnostics~\cite{Schneidewind2006} a quality metric based directly on the pixel values of a visualization. Some metrics~\cite{Bertini2006, Cui2006, Yang2003} focus on the level of abstraction, including aggregation, clustering, and sampling in these chart types. For more information on quality metrics for visualization, consult the survey by Bertini et al.~\cite{Bertini2011}. 

Some visual analytic systems leverage these metrics to recommend visualizations to their users. For example, ScagExplorer~\cite{Dang2014} applies scagnostics to cluster and filter through large collections of bivariate relationships automatically.
EvoGraphDice~\cite{Boukhelifa2013} uses evolutionary algorithms and a scagnostics-based fitness function to select interesting linear and non-linear 2D projections.
AutoVis~\cite{Wills2010} applies scagnostics to make decisions about what to visualize to provide users a first glance at their data, highlighting patterns that statisticians notice when investigating datasets.

%Our work follows their philosophy of providing guidance for data exploration but focused on the problem of selecting conditioning dimensions that reveal structure in a user-selected data relationship. 


\subsection{Non-parametric approaches in visualization}
We use a non-parametric permutation test to rank small multiple displays. Other researchers have explored other uses for non-parametric statistical methods in visualizations.

Graphical inference~\cite{Buja2009, Wickham2013, Majumder2013} asks viewers to judge whether a visualization of the actual dataset is visually distinguishable from random bootstrapped samples. The result is a non-parametric significance test of a visual pattern. Similarly, Menjoge~\cite{Menjoge2010} uses bootstrapping to generate a 95\% visual confidence interval that can correctly communicate the sampling variablity in a visual pattern. %These are interesting applications of bootstrap methods~\cite{Efron1979} that are used extensively in estimating statistical measures of accuracy when analytic methods are too expensive. However, they assume that dataset under study is a sample from an unknown population while we consider our dataset a complete representation of our population.

% I wasn't able to access this paper online. I can't tell from the abstract or the following paragraph if this approach uses anything non-parametric. Is this in the right section? 
%Multivariate Visual Explanations (MVE)~\cite{Barlowe2008} explicitly reveal the hidden multivariate relationships in a simple manner to fill the WorldView gap~\cite{Amar2004} in visualization tools that fail to provide support for the discovery of useful correlative relationships in multivariate data. MVE tightly integrates partial derivatives computation and visual inspection to reveal multivariate correlations and as the structure of interest. We investigate a general approach to visual explanations that can be used to discover various structures of interest specified by quantitative data visualization quality measures.


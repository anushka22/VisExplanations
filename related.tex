\section{Related Work}

A common strategy for visual exploration and analysis of multidimensional datasets is to apply dimensionality reduction techniques like multidimensional scaling, factor analysis and cluster analysis~\cite{Yang2003}. 

\subsection{Scagnostics}

Graph-theoretic scagnostics~\cite{Wilkinson2005}, Scagnostic distributions~\cite{Wilkinson2008}, ScagExplorer~\cite{Dang2014}

Experimental evaluation~\cite{Lehmann2015} of scagnostics and other metrics, showing that scagnostics best matches human rankings.

AutoVis~\cite{Wills2010}

\subsection{Other metrics}

Classified scatterplots: Class consistency~\cite{Sips2009} (does the scatterplot separate clusters in 2D?), Taxonomy of Visual Cluster Separation~\cite{Sedlmair2012}

Quality metrics for Radviz, Pixel-Oriented Displays, and Table Lenses~\cite{Albuquerque2010}

Rank-by-feature~\cite{Seo2005} (rank all 2D projections). a follow up~\cite{Piringer2008} that claims they do rank-by-feature on subsets and compares subsets. I thought this would be closely related to our work, but I can't figure out what the paper is doing.

Evolutionary selection~\cite{Boukhelifa2013} of linear and nonlinear 2D projections.

\subsection{Other work}
Buja et al's work on visual null hypothesis testing~\cite{Buja2009}.

"New Procedures for Visualizing Data and Diagnosing Regression Models" by R Menjoge. This is a thesis. Chapter 2 has a very interesting approach using bootstrapping to get a 95\% visual confidence interval for a single plot. It's like the effect size version of Buja's null hypothesis approach. I can't find a published paper other than the thesis.

\section{Related Work}
\label{sec:related}
Here we summarize previous work in three areas related to our work: small multiples, visual inference methods and quality metrics for data visualizations.

\subsection{Small Multiples}
Faceting is a slicing operator~\cite{Wilkinson2005GG, Munzner2014} to split up a dataset into subsets that are examined together. Often these subsets are determined by discrete values of a categorical dimension or bins of a quantitative dimension. The result of faceting is visually expressed as small multiples, which are tables of simple views directly depicting comparisons across subsets of the data. They increase the number of dimensions that can be easily visually processed and are applied in visual data analysis tools across different application domains such as geography \cite{Guo2006, Maceachren2003} and medicine \cite{Lunzer2010, Sarni2005}. Small multiple or trellis displays are also used to analyze modeling data from designed experiments~\cite{Fuentes2011}.
Furthermore, visual analysis tools such as the ggplot2 library in the R language \cite{Wickham2006} and Tableau \cite{Stolte2002} allow users to rapidly generate small multiple visualizations to explore data. However, with limited prior knowledge about the interaction effects of dimensions in a dataset, finding patterns such as repetition, change, conditional dependence with values of other factors using these views is non-trivial. Trellis displays are used to analyze modeling data from designed experiments
Tableau's Show Me \cite{mackinlay2007} suggests heuristics to lay out effective small multiples based on the data types of dimensions and functional dependencies between dimensions. However, they do not suggest a method for selecting a conditioning dimension, while that is our focus. 

\subsection{Visual Inference Methods}
Bridging the gap between exploratory and confirmatory statistics is work that investigates statistical significance testing in the hypothesis testing of visual findings~\cite{Wickham2013, Majumder2013}. Human subjects are asked whether the observed dataset looks anything like random bootstrapped samples in lineup or Rorschach protocols to enable simulation based statistical inference of visual patterns~\cite{Buja2009}. Similarly, Menjoge~\cite{Menjoge2010} uses bootstrapping to show sampling variability of a plot to get a 95\% visual confidence interval for a single plot. These are interesting applications of bootstrap methods~\cite{Efron1979} that are used extensively in estimating statistical measures of accuracy when analytic methods are too expensive. However, they assume that dataset under study is a sample from an unknown population while we consider our dataset a complete representation of our population.

Multivariate Visual Explanations (MVE)~\cite{Barlowe2008} explicitly reveal the hidden multivariate relationships in a simple manner to fill the WorldView gap~\cite{Amar2004} in visualization tools that fail to provide support for the discovery of useful correlative relationships in multivariate data. MVE tightly integrates partial derivatives computation and visual inspection to reveal multivariate correlations and as the structure of interest. We investigate a general approach to visual explanations that can be used to discover various structures of interest specified by quantitative data visualization quality metrics.

\subsection{Data Visualization Quality Metrics}
A large number of quality metrics are used in methods for high-dimensional data analysis \cite{Bertini2011}.  Many metrics for scatterplots determine projections of the data to be displayed, often for particular tasks: cluster separation~\cite{Sedlmair2012, Tatu2009}, class consistency and separation~\cite{Sips2009, Schafer2013}, interesting visual shapes~\cite{Wilkinson2005} or statistical properties~\cite{Kandel2012, Seo2005, Piringer2008}. Metrics for parallel coordinate plots~\cite{Ankerst1998, Dasgupta2010, Johansson2009, Yang2003} also focus on the ordering of dimensions. Some metrics~\cite{Bertini2006, Cui2006} focus on the level of abstraction, including aggregration and sampling, in these chart types. Others~\cite{Albuquerque2010, Ankerst1998, Schneidewind2006, Yang2003} offer metrics for radial displays, pixel maps, table lens and other visualization types. 

Evolutionary selection~\cite{Boukhelifa2013} of linear and nonlinear 2D projections.

Scagnostics non-parametrically captured aspects of trend, shape and density with a small set of measures that were analyzed to be relatively orthogonal and with simple distributions~\cite{Wilkinson2008}. ScagExplorer~\cite{Dang2014} applied scagnostics to cluster and filter through large collections of bivariate relationships automatically. AutoVis~\cite{Wills2010} applied scagnostics to make decisions about what to visualize to provide users a first glance at their data, highlighting patterns that statisticians notice when investigating datasets. Our work follows their philosophy in terms of providing guidance for data exploration in the case of selecting conditioning dimensions that reveal structure in a data relationship. 


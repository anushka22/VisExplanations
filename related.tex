\section{Previous Work}
\label{sec:related}
Our work draws on previous work in three areas of visualization---small multiple displays, cognostics, and the use of non-parametric statistics to improve visualization.

\subsection{Small Multiple Displays}
Small multiple displays are tables of similar visualizations, where each cell visualizes a subset of the data. Such displays allow viewers to make visual inferences about the conditional impact of the partitioning variable(s). Use of small multiples dates back to the work of economist W. S. Jevons in the 19th century~\cite{Kelley1973} who used them to transform tables of time series data into rich graphical displays. 

Today, many popular visual analysis tools can generate small multiple displays, such as the Trellis package in S-Plus~\cite{Becker1996-manual}, the ggplot2 library for the R language~\cite{Wickham2006}, based on Wilkinson's Grammar of Graphics~\cite{Wilkinson2005}, and the Polaris system (now Tableau)~\cite{Stolte2002}. These systems allow users to rapidly generate small multiple displays to explore their data.
Mackinlay's APT system~\cite{mackinlay1986} and Tableau's Show Me system~\cite{mackinlay2007} implement heuristics for automatically laying out effective small multiple displays based on the data types and functional dependencies in a data set.
These tools all require users to manually select the partitioning variables.

%Small multiple displays have been effectively used in a wide range of analytic domains including geography~\cite{Guo2006, Maceachren2003}, medicine~\cite{Lunzer2010, Sarni2005}, and analysis of designed experiments~\cite{Fuentes2011}.

Small multiples can also be employed as a visual layout metaphor in user interfaces for exploring the input space of encoding parameters as in Design Galleries \cite{marks1997}. The alternating use of small multiples together with a large single view has also been used an an interaction device for data exploration \cite{van2013}.

\subsection{Cognostics}

Cognostics and scagnostics were first proposed by Tukey and Tukey~\cite{Tukey1982, Tukey1985}. More computationally efficient scagnostics were proposed by Wilkinson et al.~\cite{Wilkinson2005, Wilkinson2008}. Other scagnostics have been developed to capture properties such as cluster separation~\cite{Sedlmair2012, Tatu2009}, class consistency and separation~\cite{Sips2009, Schafer2013}, or statistically-motivated measures~\cite{Kandel2012, Seo2005, Piringer2008}.

Cognostics have also been developed for other plot types. Many authors have suggested measures for parallel coordinate plots~\cite{Ankerst1998, Dasgupta2010, Johansson2009, Yang2003}. Albuquerque et al.~\cite{Albuquerque2010} offer quality measures for radial visualizations, pixel-oriented displays, and table lenses. Schneidewind et al.~\cite{Schneidewind2006} propose Pixnostics, a quality measure based directly on the pixel representation of a visualization. Some measures~\cite{Bertini2006, Cui2006, Yang2003} focus on the level of abstraction, including aggregation, clustering, and sampling in various chart types. For more information on quality measures for visualization, consult the survey by Bertini et al.~\cite{Bertini2011}. 

Some visual analytic systems leverage these measures to recommend visualizations to their users. For example, ScagExplorer~\cite{Dang2014} applies scagnostics to cluster and filter through large collections of bivariate relationships automatically.
EvoGraphDice~\cite{Boukhelifa2013} uses evolutionary algorithms and a scagnostics-based fitness function to select interesting linear and non-linear 2D projections.
AutoVis~\cite{Wills2010} applies scagnostics to make decisions about what to visualize to provide users a first glance at their data, highlighting patterns that statisticians notice when investigating datasets.
MacEachren et al.~\cite{Maceachren2003} use conditional entropy to identify pairs of variables in a high-dimensional dataset that are likely to display interesting relationships. These variables are displayed in a matrix of view types. Trelliscope~\cite{Hafen2013} uses scagnostics to organize and filter the large number of panels that result from using a trellis display on complex data.

%It leverages user-specified cognostics~\cite{Tukey1982,Tukey1985}, ``computer guiding diagnostics" or measures of the usefulness of a data view, to sample, filter and sort panels that are potentially interesting to a user. 


\subsection{Non-parametric approaches in visualization}
%We use a non-parametric permutation test to rank small multiple displays. Researchers have explored other uses for non-parametric statistical methods in visualizations.

Graphical inference~\cite{Buja2009, Wickham2013, Majumder2013} asks viewers to judge whether a visualization of the actual dataset is visually distinguishable from random bootstrapped samples. The result is a non-parametric significance test of a visual pattern. Conversely, Menjoge~\cite{Menjoge2010} uses bootstrapping to generate a 95\% visual confidence interval that can correctly communicate the sampling variability in a visual pattern. 


\section{Related Work}
\label{sec:related}
In this section we summarize previous work in three areas related to our work. We first discuss previous work on small multiple displays. We then discuss various metrics that have been proposed for measuring the quality or usefulness of particular visualizations. Our approach leverages such metrics to select good small multiple displays. Finally, we describe other applications of non-parameteric statistics to visualization.

\subsection{Small Multiple Displays}
Small multiple displays are tables of similar visualizations, where each cell visualizations a subset of the data. Such displays support visual comparisons across subsets of the data; allowing viewers to make visual inferences about the conditional impact of the partitioning variable(s).

(Need some history of small multiples references here.)

Many popular visual analysis tools support small multiple displays, including the Trellis package in S-Plus~\cite{}, the ggplot2 library in the R language~\cite{Wickham2006}, and the Polaris system~\cite{Stolte2002} (now Tableau). These systems allow users to rapidly generate small multiple visualizations to explore data. However, with limited prior knowledge about the interaction effects of dimensions in a dataset, finding visual patterns of interest such as repetition, change, and conditional dependence using these views is non-trivial.

Mackinlay's APT system~\cite{} and Tableau's Show Me system~\cite{mackinlay2007} suggests heuristics to lay out effective small multiples based on the data types and functional dependencies of user selected dimensions. However, they do not suggest a method for automatically picking dimensions for the user. 

(I'm not sure that this paragraph is necessary.)
Small multiple displays have been effectively used in a wide range of analytic domains including geography~\cite{Guo2006, Maceachren2003}, medicine~\cite{Lunzer2010, Sarni2005}, and analysis of designed experiments~\cite{Fuentes2011}.

(Need to make sure we've cited any small multiple display work from the last 3 or 4 years in InfoVis.)

%Faceting is a slicing operator~\cite{Wilkinson2005GG, Munzner2014} to split up a dataset into subsets that are examined together. Often these subsets are determined by discrete values of a categorical dimension or bins of a quantitative dimension.

\subsection{Data Visualization Quality Measures}
(This section needs an introductory sentence or two. How does this work relate to what we're going to present?)

Filtering the large number of views of a high-dimensional dataset motivated Tukey's proposal of \textit{cognostics}~\cite{Tukey1982,Tukey1985} - diagnostic measures to evaluate the usefulness of views - so users would only manually investigate a small set of high-ranked, potentially useful views. A large number of such quality metrics for high-dimensional data analysis have been surveyed in~\cite{Bertini2011}.  Many metrics for scatterplots determine projections of the data to be displayed, often for particular tasks: cluster separation~\cite{Sedlmair2012, Tatu2009}, class consistency and separation~\cite{Sips2009, Schafer2013}, interesting visual shapes~\cite{Wilkinson2005} or statistical properties~\cite{Kandel2012, Seo2005, Piringer2008}. Metrics for parallel coordinate plots~\cite{Ankerst1998, Dasgupta2010, Johansson2009, Yang2003} also focus on the ordering of dimensions. Some metrics~\cite{Bertini2006, Cui2006} focus on the level of abstraction, including aggregation and sampling, in these chart types. Others~\cite{Albuquerque2010, Ankerst1998, Schneidewind2006, Yang2003} offer metrics for radial displays, pixel maps, table lens and other visualization types. 
	
Scagnostics~\cite{Wilkinson2005} have been used most frequently in techniques that guide user exploration towards interesting views of a dataset. The set of nine scagnostic measures non-parametrically capture aspects of trend, shape and density about bivariate relationships and have proven statistical properties with simple distributions~\cite{Wilkinson2008}. ScagExplorer~\cite{Dang2014} applied scagnostics to cluster and filter through large collections of bivariate relationships automatically. EvoGraphDice~\cite{Boukhelifa2013} uses evolutionary algorithms and a scagnostics-based fitness function to select interesting linear and non-linear 2D projections. AutoVis~\cite{Wills2010} applied scagnostics to make decisions about what to visualize to provide users a first glance at their data, highlighting patterns that statisticians notice when investigating datasets. Our work follows their philosophy of providing guidance for data exploration but focused on the problem of selecting conditioning dimensions that reveal structure in a user-selected data relationship. 

\subsection{Non-parametric approaches in Visualization}
(Same here. At a high-level, how does this work relate to what we are doing? For example, ``Other researchers have explored using non-parametric statistical methods in visualizations.'')

Bridging the gap between exploratory and confirmatory statistics is work that investigates statistical significance testing in the hypothesis testing of visual findings~\cite{Wickham2013, Majumder2013}. Human subjects are asked whether the observed dataset looks anything like random bootstrapped samples in lineup or Rorschach protocols to enable simulation based statistical inference of visual patterns~\cite{Buja2009}. Similarly, Menjoge~\cite{Menjoge2010} uses bootstrapping to show sampling variability of a plot to get a 95\% visual confidence interval for a single plot. These are interesting applications of bootstrap methods~\cite{Efron1979} that are used extensively in estimating statistical measures of accuracy when analytic methods are too expensive. However, they assume that dataset under study is a sample from an unknown population while we consider our dataset a complete representation of our population.

Multivariate Visual Explanations (MVE)~\cite{Barlowe2008} explicitly reveal the hidden multivariate relationships in a simple manner to fill the WorldView gap~\cite{Amar2004} in visualization tools that fail to provide support for the discovery of useful correlative relationships in multivariate data. MVE tightly integrates partial derivatives computation and visual inspection to reveal multivariate correlations and as the structure of interest. We investigate a general approach to visual explanations that can be used to discover various structures of interest specified by quantitative data visualization quality measures.


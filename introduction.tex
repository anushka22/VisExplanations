%% \section{Introduction} %for journal use above \firstsection{..} instead

%Exploratory Data Analysis (EDA), championed by Tukey~\cite{Tukey1977}, relies heavily on visual representations in the search for informative and potentially surprising structure in data. Such analysis usually starts with an overview of the data dimensions of interest and follows a path of progressive refinement getting more focused or detailed based on the question being asked.

Understanding multidimensional data sets is a prevelant challenge in Exploratory Data Analysis~\cite{Tukey1977}. Many techniques have been proposed for visualizing multidimensional data in 2D. Perhaps the two most common techniques are
\emph{projective displays}, such as scatterplot matrices (SPLOMS), which display one or more 2D projections of the data set,
and \emph{small multiple displays} (also called collections or trellis displays)~\cite{Bertin1983, tufte1986, Becker1996}, which show 2D slices of data sets created by conditioning on one or more dimensions in the data set,

Unfortunately, as the number of dimensions in the data set grows, neither approach scales well since the number of plots that must be displayed increases quickly. This problem can be addressed by selecting a subset of the dimensions to project or condition on. However, in exploratory analysis scenarios, where the user does not know a priori which dimensions might be of interest, this can become a time-consuming exercise in trial and error as the user manually iterates through dimensions to find views that help explain their data set. 

In the context of projective displays, there has been substantial work in developing algorithms to automate this effort~\cite{Seo2005,Wilkinson2005,Sips2009}. Perhaps the most well-known is Scagnostics which....
Automated dimensionality reduction techniques~\cite{Friedman1974,Yang2003,Sedlmair2013}, developed in statistics and machine learning, can also be used, though the resulting visualizations can be difficult to interpret since the axes may not be meaningful to users.

However, there has been little corresponding work in the automatic selection of conditioning dimensions for small multiple displays. 
In this paper we address this problem.
We assume that a data analyst has already chosen a visualization that shows a visual pattern of interest (as shown in Figure~1(a)). The analyst is interested in understanding this pattern further by conditioning on other variables in the data set. We want to find a way to automatically suggest conditioning variables that are likely to help the user understand the pattern they see (e.g. Figure~1(b)).

We propose a method of selecting a Partitioner dimension to facet a given bivariate relationship by using significance tests of a particular score metric of the resulting splits. Our contributions are:
\begin{itemize}
    \item A set of goodness criteria for the collection of splits resulting from faceting with a particular Partitioner dimension.
    \item A method for quantitatively evaluating the quality of the splits. We compare the score metric against  reference distributions of the metric from bootstrapped random sample splits which act as ``null splits".
\end{itemize}

The next section summarizes related work on visual explanations for multivariate data analysis. This is followed by a description of the method we propose and discussion of metrics we use. Then we describe examples using our method to explain high-dimensional structure in a number of datasets. Finally, we draw conclusions from this research and outline future work.


Move to related work?

Faceting is a slicing operator~\cite{Wilkinson2005GG, Munzner2014} to split up a dataset into subsets that are examined together. Often these subsets are determined by discrete values of a categorical dimension or bins of a quantitative dimension. The result of faceting is expressed visually as \textit{Small Multiples}~\cite{tufte1986} or \textit{Collections}\cite{Bertin1983} and \textit{Trellis} displays \cite{Becker1996}. Small multiples facilitate finding structure and patterns in complex data by forming tables of simple views directly depicting comparisons across dimensions in the data. They increase the number of dimensions that can be easily visually processed and are applied in visual data analysis tools across different application domains such as geography \cite{Guo2006, Maceachren2003} and medicine \cite{Lunzer2010, Sarni2005}. Furthermore, visual analysis tools such as the ggplot2 library in the R language \cite{Wickham2006} and Tableau \cite{Stolte2002} allow users to rapidly generate small multiple visualizations to explore data. However, with limited prior knowledge about the interaction effects of dimensions in a dataset, it becomes an exercise of trial and error to find the \textit{Partitioner} dimension that reveals ``interesting" structure in the subsets resulting from faceting.


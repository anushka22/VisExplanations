%% \section{Introduction} %for journal use above \firstsection{..} instead

Exploratory Data Analysis (EDA), championed by Tukey~\cite{Tukey1977}, relies heavily on visual representations in the search for informative and potentially surprising structure in data. Such analysis usually starts with an overview of the data dimensions of interest and follows a path of progressive refinement getting more focused or detailed based on the question being asked. A common strategy for visual exploration and analysis of multidimensional datasets is to examine two-dimensional projections often through selecting ``interesting" axis-aligned projections~\cite{Seo2005,Wilkinson2005,Sips2009}. Another popular approach is examining axis-unaligned projections through dimensionality reduction techniques~\cite{Friedman1974,Yang2003,Sedlmair2013}. However, there is little work in exploring how to slice two-dimensional projections into multiple plots with ``interesting" visual structure. Here we propose a methodology, applying the principle of statistical significance testing, to determine how to select such slices. 
 
Faceting is a slicing operator~\cite{Wilkinson2005GG, Munzner2014} to split up a dataset into subsets that are examined together. Often these subsets are determined by discrete values of a categorical dimension or bins of a quantitative dimension. The result of faceting is expressed visually as \textit{Small Multiples}~\cite{tufte1986} or \textit{Collections}\cite{Bertin1983} and \textit{Trellis} displays \cite{Becker1996}. Small multiples facilitate finding structure and patterns in complex data by forming tables of simple views directly depicting comparisons across dimensions in the data. They increase the number of dimensions that can be easily visually processed and are applied in visual data analysis tools across different application domains such as geography \cite{Guo2006, Maceachren2003} and medicine \cite{Lunzer2010, Sarni2005}. Furthermore, visual analysis tools such as the ggplot2 library in the R language \cite{Wickham2006} and Tableau \cite{Stolte2002} allow users to rapidly generate small multiple visualizations to explore data. However, with limited prior knowledge about the interaction effects of dimensions in a dataset, it becomes an exercise of trial and error to find the \textit{Partitioner} dimension that reveals ``interesting" structure in the subsets resulting from faceting.

We propose a method of selecting a Partitioner dimension to facet a given bivariate relationship by using significance tests of a particular score metric of the resulting splits. Our contributions are:
\begin{itemize}
    \item A set of goodness criteria for the collection of splits resulting from faceting with a particular Partitioner dimension.
    \item A method for quantitatively evaluating the quality of the splits. We compare the score metric against  reference distributions of the metric from bootstrapped random sample splits which act as ``null splits".
\end{itemize}

The next section summarizes related work on visual explanations for multivariate data analysis. This is followed by a description of the method we propose and discussion of metrics we use. Then we describe examples using our method to explain high-dimensional structure in a number of datasets. Finally, we draw conclusions from this research and outline future work.




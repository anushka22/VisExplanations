\section{Discussion}
\label{sec:discussion}
(Answer potential reviewer's question about how this relates to stepwise regression type methods + feature selection in other methods)
Statistical methods like ridge and lasso regression help automatically select subsets of variables that produce good explanatory models of a set of multivariate observations. Using lasso regression to progressively add explanatory variables assumes interest in a linear model and may not translate to different, visually interesting  patterns in consecutive steps. These methods build up joint models while we are look at pairwise models considering adding one explanatory variable independent of the rest. 

Non-parametric approaches allow us to use quality metrics without a closed form distribution, which seems almost essential in evaluating visual patterns. However, these approaches are computational demanding due to the need to recompute the metric on a large number of samples. We need more work on efficient visual metrics that are robust to the size of samples.

A possible other criteria for a good small multiple display is one that minimizes the redundancy between partitioning variables and selected variables in the original visualization. While exposing highly correlated variables can be useful, it is likely not what the user wants in an effective small multiple display. We need a way to detect this. (NEED an idea/hypothesis here)

We could develop a decision tree based exploratory data analysis interaction mechanism guided by our algorithm. At each decision level, we could apply our algorithm to select a partitioning variable given a single view of the data at that level. This would produce a small multiple display where each component plot could be further partitioned to reveal interesting structure. Considering the tree structure, each choice of a partitioning variable would be conditional on the other previously used variables, as in model selection methods. 

Another key decision in creating small multiples [of aggregated data?] is whether to use a particular variable to partition the original view or to add to the level of detail (roll-up/drill-down). We could employ our approach using permutation tests  to determine whether the visual pattern seen in a view is one due to chance or interesting given the selected level of detail. (NEED better articulation).



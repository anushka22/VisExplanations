%\documentclass[journal]{vgtc}                % final (journal style)
\documentclass[review,journal]{vgtc}         % review (journal style)
%\documentclass[widereview]{vgtc}             % wide-spaced review
%\documentclass[preprint,journal]{vgtc}       % preprint (journal style)
%\documentclass[electronic,journal]{vgtc}     % electronic version, journal

%% Uncomment one of the lines above depending on where your paper is
%% in the conference process. ``review'' and ``widereview'' are for review
%% submission, ``preprint'' is for pre-publication, and the final version
%% doesn't use a specific qualifier. Further, ``electronic'' includes
%% hyperreferences for more convenient online viewing.

%% Please use one of the ``review'' options in combination with the
%% assigned online id (see below) ONLY if your paper uses a double blind
%% review process. Some conferences, like IEEE Vis and InfoVis, have NOT
%% in the past.

%% Please note that the use of figures other than the optional teaser is not permitted on the first page
%% of the journal version.  Figures should begin on the second page and be
%% in CMYK or Grey scale format, otherwise, colour shifting may occur
%% during the printing process.  Papers submitted with figures other than the optional teaser on the
%% first page will be refused.

%% These three lines bring in essential packages: ``mathptmx'' for Type 1
%% typefaces, ``graphicx'' for inclusion of EPS figures. and ``times''
%% for proper handling of the times font family.

\usepackage{mathptmx}
\usepackage{graphicx}
\usepackage{times}
\usepackage{url}
\usepackage{graphics}
\usepackage{subcaption}
\captionsetup[subfigure]{labelformat=simple}
\renewcommand\thesubfigure{(\alph{subfigure})}

%% We encourage the use of mathptmx for consistent usage of times font
%% throughout the proceedings. However, if you encounter conflicts
%% with other math-related packages, you may want to disable it.

%% This turns references into clickable hyperlinks.
\usepackage[bookmarks,backref=true,linkcolor=black]{hyperref} %,colorlinks
\hypersetup{
  pdfauthor = {},
  pdftitle = {},
  pdfsubject = {},
  pdfkeywords = {},
  colorlinks=true,
  linkcolor= black,
  citecolor= black,
  pageanchor=true,
  urlcolor = black,
  plainpages = false,
  linktocpage
}

% Alter some LaTeX defaults for better treatment of figures:
% See p.105 of "TeX Unbound" for suggested values.
% See pp. 199-200 of Lamport's "LaTeX" book for details.
% General parameters, for ALL pages:
\renewcommand{\topfraction}{0.9}    % max fraction of floats at top
\renewcommand{\bottomfraction}{0.8} % max fraction of floats at bottom
% Parameters for TEXT pages (not float pages):
\setcounter{topnumber}{2}
\setcounter{bottomnumber}{2}
\setcounter{totalnumber}{4} % 2 may work better
\setcounter{dbltopnumber}{1} % for 2-column pages
\renewcommand{\dbltopfraction}{0.9} % fit big float above 2-col. text
\renewcommand{\textfraction}{0.05}  % allow minimal text w. figs
% Parameters for FLOAT pages (not text pages):
\renewcommand{\floatpagefraction}{0.7}  % require fuller float pages
% N.B.: floatpagefraction MUST be less than topfraction !!
\renewcommand{\dblfloatpagefraction}{0.7}   % require fuller float pages

% remember to use [htp] or [htpb] for placement
 
%% If you are submitting a paper to a conference for review with a double
%% blind reviewing process, please replace the value ``0'' below with your
%% OnlineID. Otherwise, you may safely leave it at ``0''.
\onlineid{193}

%% declare the category of your paper, only shown in review mode
\vgtccategory{Theory/Model}

%% allow for this line if you want the electronic option to work properly
\vgtcinsertpkg

%% In preprint mode you may define your own headline.
%\preprinttext{To appear in IEEE Transactions on Visualization and Computer Graphics.}

%% Paper title.

\title{Automatic Selection of Partitioning Variables\\for Small Multiple Displays}

%% This is how authors are specified in the journal style

%% indicate IEEE Member or Student Member in form indicated below
\author{Anushka Anand and Justin Talbot}
\authorfooter{
%% insert punctuation at end of each item
\item
Anushka Anand is with Tableau Research. E-mail: aanand@tableau.com.
\item
Justin Talbot is with Tableau Research. E-mail: jtalbot@tableau.com.
 }

%other entries to be set up for journal
\shortauthortitle{Anand \MakeLowercase{\textit{et al.}}: Automatic Selection of Partitioning Variables\\for Small Multiple Displays}
%\shortauthortitle{Firstauthor \MakeLowercase{\textit{et al.}}: Paper Title}

%% Abstract section.
\abstract{
%Small multiple displays are a common approach for visualizing multidimensional data.
Effective small multiple displays are created by partitioning a visualization on variables that reveal interesting conditional structure in the data.
%If the number of variables in the data set is large, or if the analyst has limited prior information about the data set, choosing an informative partitioning variable can require substantial trial and error.
We propose a method that automatically ranks the possible partitioning variables in a data set, allowing analysts to focus on the most promising small multiple displays.
Our approach is based on a randomized, non-parametric permutation test, which allows us to handle a wide range of quality measures for visual patterns defined on many different visualization types, while discounting spurious patterns.
We demonstrate the effectiveness of our approach on scatterplots of real-world, multidimensional datasets.
} % end of abstract

%% Keywords that describe your work. Will show as 'Index Terms' in journal
%% please capitalize first letter and insert punctuation after last keyword
\keywords{Small multiple displays, Visualization selection, Multidimensional data.}

%% ACM Computing Classification System (CCS). 
%% See <http://www.acm.org/class/1998/> for details.
%% The ``\CCScat'' command takes four arguments.

%\CCScatlist{ % not used in journal version
 %\CCScat{K.6.1}{Management of Computing and Information Systems}%
%{Project and People Management}{Life Cycle};
 %\CCScat{K.7.m}{The Computing Profession}{Miscellaneous}{Ethics}
%}

%% Uncomment below to include a teaser figure.
\teaser{
 \centering 
	 \begin{subfigure}{2in}
 		\includegraphics[height=2in]{images/teaser1.pdf}
        \vspace{-0.3in}
		\caption{Input scatterplot}
		 \label{fig:teaser1}
	 \end{subfigure}
	 \begin{subfigure}{5in}
 		\includegraphics[height=2in]{images/teaser2.pdf}
        \vspace{-0.3in}
		\caption{Output small multiple display partitioned on admit ACT scores}
		 \label{fig:teaser2}
	 \end{subfigure}
   \caption{An example of our approach. On the left is an input plot showing the relationship between admission rates and graduation rates at US universities. On the right is the top ranked small multiple display automatically picked by our algorithm to help explain the input data. It partitions the data on admit ACT scores, revealing that for universities with very high average ACT scores, there is a strong linear relationship between the two selected variables, while for other universities, there is no clear relationship.}
 \label{fig:teaser}
}

%% Uncomment below to disable the manuscript note
%\renewcommand{\manuscriptnotetxt}{}

%% Copyright space is enabled by default as required by guidelines.
%% It is disabled by the 'review' option or via the following command:
% \nocopyrightspace

%%%%%%%%%%%%%%%%%%%%%%%%%%%%%%%%%%%%%%%%%%%%%%%%%%%%%%%%%%%%%%%%
%%%%%%%%%%%%%%%%%%%%%% START OF THE PAPER %%%%%%%%%%%%%%%%%%%%%%
%%%%%%%%%%%%%%%%%%%%%%%%%%%%%%%%%%%%%%%%%%%%%%%%%%%%%%%%%%%%%%%%%

\begin{document}

%% The ``\maketitle'' command must be the first command after the
%% ``\begin{document}'' command. It prepares and prints the title block.

%% the only exception to this rule is the \firstsection command
\firstsection{Introduction}

\maketitle

%% \section{Introduction} %for journal use above \firstsection{..} instead

%Exploratory Data Analysis (EDA), championed by Tukey~\cite{Tukey1977}, relies heavily on visual representations in the search for informative and potentially surprising structure in data. Such analysis usually starts with an overview of the data dimensions of interest and follows a path of progressive refinement getting more focused or detailed based on the question being asked.

Understanding multidimensional data sets is a prevalent challenge in Exploratory Data Analysis~\cite{Tukey1977}. Many techniques have been proposed for visualizing multidimensional data in 2D. Perhaps the two most common techniques are
\emph{projective displays}, such as scatterplot matrices (SPLOMS), which display one or more 2D projections of the data set,
and \emph{small multiple displays} (also called collections or trellis displays)~\cite{Bertin1983, tufte1986, Becker1996}, which show 2D slices of data sets created by conditioning on one or more dimensions in the data set,

Unfortunately, as the number of dimensions in the data set grows, neither approach scales well since the number of plots that must be displayed increases quickly. This problem can be addressed by selecting a subset of the dimensions to project or condition on. However, in exploratory analysis scenarios, where the user does not know a priori which dimensions might be of interest, this can become a time-consuming exercise in trial and error as the user manually iterates through dimensions to find views that help explain their data set. 

In the context of projective displays, there has been substantial work in developing algorithms to automate this effort~\cite{Seo2005,Wilkinson2005,Sips2009}. Perhaps the most well-known is Scagnostics which are a set of graph-theoretic, diagnostic measures that non-parametrically characterize the distributional shape of 2D point clouds and are used to rank or select 2D projections. Automated dimensionality reduction techniques~\cite{Friedman1974,Yang2003,Sedlmair2013}, developed in statistics and machine learning, can also be used, though the resulting visualizations can be difficult to interpret since the axes, often linear combinations of dimensions, may not be meaningful to users.

However, there has been little corresponding work in the automatic selection of conditioning dimensions for small multiple displays. In this paper we address this problem.
We assume that a data analyst has already chosen a visualization that shows a visual pattern of interest (as shown in Figure~1(a)). The analyst is interested in understanding this pattern further by conditioning on other variables in the data set. We want to find a way to automatically suggest conditioning variables that are likely to help the user understand the pattern they see (e.g. Figure~1(b)). 

We propose a method of selecting a dimension to condition a given data relationship by using non-parametric permutation tests to determine the significance of the resulting collection of data partitions. Our contributions are:
\begin{itemize}
    \item A set of goodness criteria for the collection of partitions given a particular conditioning dimension.
    \item A method for quantitatively evaluating the quality of the partitions given a measure of interest. We compare the measure computed on a collection of partitions against reference distributions of the measure computed on randomly generated permutations of partitions which act as ``null partitions". 
\end{itemize}

The next section summarizes related work on visual explanations for multivariate data analysis. This is followed by a description of the method we propose and discussion of measures we use. Then we describe examples using our method to explain high-dimensional structure in a number of datasets. Finally, we draw conclusions from this research and outline future work.


\section{Previous Work}
\label{sec:related}
Our work draws on previous work in three areas of visualization---small multiple displays, cognostics, and the use of non-parametric statistics to improve visual analytics.

\subsection{Small Multiple Displays}
Small multiple displays are tables of similar visualizations, where each cell visualizes a subset of the data. Such displays allow viewers to make visual inferences about the conditional impact of the partitioning variable(s). Use of small multiples dates back to the work of economist W. S. Jevons in the 19th century~\cite{Kelley1973} who used them to transform tables of time series data into rich graphical displays. 

Today, many popular visual analysis tools can generate small multiple displays, such as the Trellis package in S-Plus~\cite{Becker1996-manual}, the ggplot2 library for the R language~\cite{Wickham2006}, based on Wilkinson's Grammar of Graphics~\cite{Wilkinson2005}, and the Polaris system (now Tableau)~\cite{Stolte2002}. These systems allow users to rapidly generate small multiple displays to explore their data.
Mackinlay's APT system~\cite{mackinlay1986} and Tableau's Show Me system~\cite{mackinlay2007} implement heuristics for automatically laying out effective small multiple displays based on the data types and functional dependencies in a data set.
These tools all require users to manually select the partitioning variables.

%Small multiple displays have been effectively used in a wide range of analytic domains including geography~\cite{Guo2006, Maceachren2003}, medicine~\cite{Lunzer2010, Sarni2005}, and analysis of designed experiments~\cite{Fuentes2011}.

Small multiples can also be employed as a visual layout metaphor in user interfaces for exploring the input space of encoding parameters as in Design Galleries \cite{marks1997}. The alternating use of small multiples together with a large single view has also been used an an interaction device for data exploration \cite{van2013}.

\subsection{Cognostics}

Cognostics and scagnostics were first proposed by Tukey and Tukey~\cite{Tukey1982, Tukey1985}. More computationally efficient scagnostics were proposed by Wilkinson et al.~\cite{Wilkinson2005, Wilkinson2008}. Other scagnostics have been developed to capture properties such as cluster separation~\cite{Sedlmair2012, Tatu2009}, class consistency and separation~\cite{Sips2009, Schafer2013}, or statistically-motivated measures~\cite{Kandel2012, Seo2005, Piringer2008}.

Cognostics have also been developed for other plot types. Many authors have suggested quality measures for parallel coordinate plots~\cite{Ankerst1998, Dasgupta2010, Johansson2009, Yang2003}. Albuquerque et al.~\cite{Albuquerque2010} offer  measures for radial visualizations, pixel-oriented displays, and table lenses. Schneidewind et al.~\cite{Schneidewind2006} propose Pixnostics, a cognostic based directly on the pixel representation of a visualization. Some measures~\cite{Bertini2006, Cui2006, Yang2003} focus on the level of abstraction, including aggregation, clustering, and sampling in various chart types. For more information on quality measures for visualization, consult the survey by Bertini et al.~\cite{Bertini2011}. 

Some visual analytic systems leverage these measures to recommend visualizations to their users. For example, ScagExplorer~\cite{Dang2014} applies scagnostics to cluster and filter through large collections of bivariate relationships automatically.
EvoGraphDice~\cite{Boukhelifa2013} uses evolutionary algorithms and a scagnostics-based fitness function to select interesting linear and non-linear 2D projections.
AutoVis~\cite{Wills2010} uses scagnostics to provide users with effective summaries of their data, tuned to highlight patterns that professional statisticians would also identify.
MacEachren et al.~\cite{Maceachren2003} use conditional entropy to identify pairs of variables in a high-dimensional dataset that are likely to display interesting relationships. These variables are displayed in a matrix of view types. Trelliscope~\cite{Hafen2013} uses scagnostics to organize and filter the large number of panels that result from using a trellis display on complex data.

%It leverages user-specified cognostics~\cite{Tukey1982,Tukey1985}, ``computer guiding diagnostics" or measures of the usefulness of a data view, to sample, filter and sort panels that are potentially interesting to a user. 


\subsection{Non-parametric statistics in visualization}
%We use a non-parametric permutation test to rank small multiple displays. Researchers have explored other uses for non-parametric statistical methods in visualizations.

Graphical inference~\cite{Buja2009, Wickham2013, Majumder2013} asks viewers to judge whether a visualization of the actual dataset is visually distinguishable from random bootstrapped samples. The result is a non-parametric significance test of a visual pattern. Conversely, Menjoge~\cite{Menjoge2010} uses bootstrapping to generate a 95\% visual confidence interval that can correctly communicate the sampling variability in a visual pattern. 


\section{Method}
\label{sec:method}

Our approach assumes that we have been given as input: (1) a user-selected visualization which the user wants to partition into a small multiple display, (2) a quality measure defined on this visualization type that captures patterns of interest to the user, and (3) the data set underlying the visualization which includes additional variables not mapped to visual variables that are potential partitioning variables in a small multiple display. The output of our method is a scoring of the small multiple displays produced by each partitioning variable.

\subsection{Goodness-of-Split Criteria}
We would like our approach to select small multiple displays that have the following four properties:
\begin{itemize}
\item \emph{Visually rich}: We want small multiple displays that convey rich visual patterns, as captured by the quality metric provided by the analyst. In contrast to statistical methods, such as ANOVA, which are based on relatively simple summary metrics with closed-form distributions, most visualization quality metrics involve complicated processing and do not follow a known distribution.

\item \emph{Informative}: The purpose of a small multiple display is to help explain patterns in the input visualization. We want our method to pick partitioning variables that are likely to be informative. Partitions that randomly split the data are not useful.

\item \emph{Well-supported}: For some data sets, particularly those with outliers or with a small number of data points, high quality scores can occur by chance. We would like to detect and downweight spurious patterns in such displays.

\item \emph{Parsimonious}: A small multiple display with many partitions can be very difficult to read and understand. All things being equal, we want to favor splitting into as few plots as possible, while still providing an informative display.
\end{itemize}

\subsection{Algorithm}

\begin{figure}
 \centering 
\includegraphics[width=1.5in]{images/age-capital_gain.pdf}
  \caption{Original interesting bivariate relationship}
 \label{fig:method_original}
\end{figure}

%Given a user-selected set of response variables, we want to automatically find a good Split respecting the criteria outlined above. This process of selecting a good Split, adding a conditioning variable to explain the visual structure in a set of data observations when we have many potential such variables, is akin to the model selection process. We evaluate the ``model" a Split determines based not only on the quality of the interesting patterns in partitions, but also, on the simplicity or size of the collection of partitions. Model selection is commonly employed in statistics and data mining to pick a statistical model that fits a sample of data, such as curve fitting or clustering. However, applying such a methodology with cognostics or visual pattern measures studied in the information visualization community has not been explored. We describe an algorithm to apply cognostics to do model selection where a model is determined as the addition of an explanatory variable to a selected set of response variables. 


Our approach is based on a permutation test, which is a non-parametric statistical significance test where the distribution of a test statistic under the null hypothesis is determined by computing the test statistic on rearrangements of the observed data points. Here, we find the statistical significance of a particular small multiple display using a cognostic as the test statistic. This allows us to evaluate small multiples for patterns not due to chance, so we can automatically find a good small multiples.

To understand how this algorithm works, consider the example in Figure~\ref{fig:method_original}. Blah blah about the dataset...

Central to our approach is the use of a cognostic that tracks the visually interesting pattern we seek to distinguish in a Split. The cognostic is an independent, substitutable piece of the approach and its choice is based on the user's task and interest. For this example, we use the Clumpy scagnostic measure~\cite{Wilkinson2005} as our cognostic.

Let the partitioning variable $d_p$ create $k$ partitions with sizes $\left\{ {s_1, s_2,...,s_k}\right\}$. We compute the cognostic on each of these $k$ partitions to get a set of interestingness measures $\left\{ {c_1, c_2,...,c_k}\right\}$. Figure~\ref{fig:method_actual} shows the small multiple resulting from using the "Relationship" variable to partition the original view. The partitions, going from the top of the bottom, have sizes $\left\{ {6523, 4278, 525, 2513,1679, 763}\right\}$  and Clumpy measures $\left\{ {0.033, 0.038, 0.086, 0.019, 0.042, 0.049}\right\}$.

\begin{figure*}
 \centering 
	 \begin{subfigure}{1.25in}
 	\centering 
  		\includegraphics[width=1.25in]{images/relationship.pdf}
		\caption{}
		 \label{fig:method_actual}
	 \end{subfigure}
	 \begin{subfigure}{1.25in}
 	\centering 
 		\includegraphics[width=1.25in]{images/randCluster.pdf}
		 \label{fig:method_random}
 		\caption{}
	 \end{subfigure}
	 \begin{subfigure}{2in}
 	\centering 
 		\includegraphics[width=2in]{images/hist-relationship.pdf}
		\caption{}
		 \label{fig:method_dist}
	 \end{subfigure}
   \caption{Illustration of our method of evaluating small multiple displays. (a) Partitions determined by the Relationship variable. (b)Randomly permuted partitions of data. (c) Distribution of Clumpy measures for randomly permuted partitions. The overlaid blue lines are the corresponding Clumpy measures of the partitions determined by Relationship in (a).}
\end{figure*}

Then, we generate a random permutation of the original data into $k$ partitions of the same sizes $\left\{ {s_1, s_2,...,s_k}\right\}$ as seen in Figure~\ref{fig:method_random}. We use permutation sampling as we assume our dataset is representative of the population and do not need to compensate, via bootstrapping, for the assumption that it represents a sample from a larger, unexamined population. (NEED MORE HERE). We compute the Clumpy measure on each of these $k$ random partitions. Then we generate $r$ such randomly permuted partitions and corresponding sets of measures to create $k$ distributions of cognostics as seen in Figure~\ref{fig:method_dist}. 

The generated cognostic distributions function as reference "null distributions" and we apply a non-parametric significance test to determine how significant the difference is between the partitions determined by a particular variable and random partitions. The z-score from Chebyshev's inequality is as follows:
$$\sum_{i=1}^n \frac{(x_i-\mu_i)^2}{\sigma_i^2}$$
where $x_i$ is the score of the i-th partition and $\mu_i$ and $\sigma_i$ are the mean and standard deviation of the cognostic measures for the i-th partition for the randomly-generated partitions. We use the z-scores to ranking partitioning variables and the Relationship variable produces the highest ranked small multiple at $9.995$ given the original view.

\subsection{Handling Continuous Partitioners}
Determining discrete splits for a categorical variable is trivial as the observations are naturally partitioned into subsets for each discrete choice the variable offers. For continuous variables, discrete partitions can be created through a binning technique. There are various binning techniques that split a continuous range into disjoint intervals~\cite{Freedman1981,Scott2009} employed in histograms. An alternative binning strategy is one with overlapping bins of roughly equal count called shingles~\cite{Becker1996 ?Cleveland book first??}. The overlap of the intervals increases the resolution with which we can study conditional independence in the same way that moving averages increase the resolution of local behavior at each time point. 
%Figure~\ref{fig:shingles} shows such �moving snapshots� of the data across the range of the partitioning variable.

\subsection{Multiple Partitioners}
Small multiples or trellis plots facet a single view into multiple views, each displaying subsets of data conditioned on the variables defining the small multiple. These conditioning or partitioning variables are combined using cross and nest operators~\cite{Wilkinson2005GG,Stolte2002} to define the layout of the small multiple. Our algorithm could be used to pick variables to partition on in the transition from a large single view~\cite{van2013} to informative small multiples.



\section{Evaluation}
\label{sec:evaluation}
Here we evaluate how our algorithm satisfies the properties we set up as the goodness-of-split criteria.

\subsection{Visually Rich}

A visualization of a data relationship appears useful when it has a visually salient pattern that translates to some simple model of the relationship, for example, a linear trend. Visual pattern measures, such as scagnostics that non-parametrically characterize the distributional shape seen in a scatterplot, can distill structure missed by ANOVA based analysis. For example, consider the bivariate relationship between linolenic and linoleic in Figure~\ref{fig:vrich_all} from the olive oils dataset~\cite{Forina1983}, which represents eight chemical measurements on different specimens of olive oil produced in various regions in Italy. There appears to be visually striking clumps and striation patterns overlaid. An analyst might wonder whether these patterns can be isolated and explained by any of the other dimensions in the dataset.
 
\begin{figure}
 \centering 
	 \begin{subfigure}{1.25in}
		\includegraphics[width=1.25in]{images/linolenic-linoleic.pdf}
		  \caption{}
		 \label{fig:vrich_all}
	\end{subfigure}
	 \begin{subfigure}{2.5in}
		\includegraphics[width=2.5in]{images/15_2321946775352-region.pdf}
		  \caption{}
		 \label{fig:vrich_sm}
	\end{subfigure}
	  \caption{(a) User-selected bivariate relationship of two chemicals in the Olive oils dataset. (b) The highest ranked small multiple on the Striation scagnostic pulls apart the striated patterns from the original bivariate relationship. }
\end{figure}

Using the Striation scagnostic, our approach ranks the "Region" variable as the best partitioning dimension for isolating such patterns. Figure~\ref{fig:vrich_sm} reveals the clean isolation of the striation pattern for olive oils from the Liguria region and the distinctive measurement structures of linoleic values for the Umbria region. This small multiple explains the visually rich patterns seen in Figure~\ref{fig:vrich_all}.

\subsection{Informative}

When a small multiple display reveals unexpected or different structure than that seen in the original view, it adds to the user's understanding of the dataset. Different visual patterns function as an indicator of the partitioner variable's explanatory power and its independence relative to the response variables being examined.

\begin{figure}
 \centering 
	 \begin{subfigure}{1.25in}
		\includegraphics[width=1.25in]{images/DEATH_RT-BIRTH_RT.pdf}
		  \caption{}
		 \label{fig:informative_all}
	\end{subfigure}
	\begin{subfigure}{3in}
		\includegraphics[width=3in]{images/6_84034106410344-URBAN.pdf}
		 \label{fig:informative_sm}
		  \caption{}
	 \end{subfigure}
	\begin{subfigure}{3in}
		\includegraphics[width=3in]{images/2_48595929670884-LEADER.pdf}
		  \caption{}
		 \label{fig:informative_sm_big}
	 \end{subfigure}
	  \caption{(a) User-selected relationship between birth and death rates for countries around the world. (b) The highest ranked small multiple display shows partitions that reveal strong opposite trends that was not seen in the original view. (c) The lowest ranked small multiple display that reveals interesting correlations but is not parsimonious.}
\end{figure}
An illustration of an informative view involves the use of the Monotonic scagnostic  which is the ranked Spearman correlation coefficient and the Ourworld dataset of UN statistics on world countries~\cite{Wilkinson2005GG,Wilkinson2008}. We want to determine how to partition the bivariate relationship between Birth rate and Death rate seen in Figure~\ref{fig:informative_all}. The highest ranked small multiple is that determined by the variable Urban that partitions the data into two categories - city with $40$ points and rural with $17$ points. This variable can be considered a confounding covariate, the unexamined field that has an effect on the data pattern. Figure~\ref{fig:informative_sm} shows that the Birth rate is negatively correlated with Death rate in cities and positively correlated in rural settings. The negative correlation is contrary to the pattern in the original view. As such it is an example of Simpson's paradox when aggregate numbers are affected by changes in the relative size and value of the subpopulations. 
 
Figure~\ref{fig:informative_sm_big} is another informative small multiple display determined by the religion of the Leader of the countries. However, this variable has more partitions and weaker patterns as each has lower support. 

\subsection{Support}

\begin{figure}
\centering
\includegraphics[width=3.25in,height=3.25in]{images/support.pdf}
  \caption{The effect of support on the combined z-scores ranking of partitioning dimensions for the data about American universities. As the number of points in the dataset decrease, the importance of the variable determined by the z-scores decrease too. }
 \label{fig:support}
\end{figure}
To examine how conservative our approach is in its assessment of interesting patterns for those partitions with low support, we conduct an experiment using the dataset about US universities. We compute the rankings based on the combined z-scores for all the partitioning dimensions in the dataset. Then we progressively remove $10\%$ of the data points at random until we barely have any points left in the dataset. At each step we recompute the rankings of the partitioning dimensions. We expect to the see the z-scores generally decreasing as the partitions have fewer points. Figure~\ref{fig:support} confirms the effect of decreasing support to the decreasing importance given to visual patterns by our approach.

\subsection{Parsimonious}
  \begin{figure}
    \centering
    \begin{minipage}[b]{1.5in}
       \begin{subfigure}[b]{\linewidth}
  	\includegraphics[width=0.65in]{images/donut1-donut2.pdf}
      \caption{}
      \label{fig:pars1}
      \end{subfigure}\\[\baselineskip]
      \begin{subfigure}[b]{\linewidth}
  	\includegraphics[width=1.5in]{images/19_7340782313668-cluster.pdf}
      \caption{}
      \label{fig:pars2}
      \end{subfigure}
      \begin{subfigure}[b]{\linewidth}
  	\includegraphics[width=1.5in]{images/8_12377386346542-cluster1.pdf}
        \caption{}
      \label{fig:pars3}        
      \end{subfigure}
    \end{minipage}
    \begin{subfigure}[b]{1.5in}
	\includegraphics[width=1.5in]{images/5_54820204216055-cluster2.pdf}
      \caption{}
      \label{fig:pars4}
    \end{subfigure}
    \caption{The ranking of small multiple displays respects the parsimony criterion. (a) The original bullseye pattern. (b) The best small multiple determined by the Clumpy scagnostic. (c) The second best partitioning dimension redundantly halves the $2$ partitions from (b). (d) The lowest ranked small multiple display with $8$ partitions.}
    \label{fig:parsimonious}
  \end{figure}

High-cardinality Partitioners create a large number splits, which likely produce splits with low support as the observations get distributed among more facets. These small multiples are expectedly penalized by our approach as seen in Figure~\ref{fig:informative_sm_big}.

We examine the ability of our method to favor parsimonious small multiples using an artificially generated dataset so we can hold the visual patterns across partitioning dimensions equal as far as possible. We take the bullseye pattern shown in Figure~\ref{fig:pars1} and have a partitioning dimension that cleanly separates the ring from the core as seen in Figure~\ref{fig:pars2}. Then we create partitioning dimensions that repeatedly randomly halve the points in the partitions from the previous step to create small multiples as seen in Figures~\ref{fig:pars3} and~\ref{fig:pars4}. We compute the ranking of these partitioning dimensions and see the small multiples in the order as shown in Figure~\ref{fig:parsimonious}. We favor the parsimonious small multiple over the redundant ones that show similar visual patterns with low support. 


\section{Discussion}
\label{sec:discussion}
(Answer potential reviewer's question about how this relates to stepwise regression type methods + feature selection in other methods)
Statistical methods like ridge and lasso regression help automatically select subsets of variables that produce good explanatory models of a set of multivariate observations. However, these methods make a number of assumptions about the errors in the model given the sample and about the independence of  variables. Using lasso regression to progressively add explanatory variables assumes interest in a linear model and may not translate to different, visually interesting  patterns in consecutive steps.

Weaknesses and future work:

When the parametric assumption (e.g. normal distribution of data) is correct, the parametric tests (z-, t-, F-tests) are more powerful.
Permutation test is more computationally intensive

Non-parametric approaches allow us to use quality metrics without a closed form distribution, which seems almost essential in evaluating visual patterns. However, these approaches are computational demanding due to the need to recompute the metric on a large number of samples. We need more work in efficient visual metrics.

redundancy between partitioning variables and other variables in the single viz. While exposing highly correlated variables can be useful, it is likely not what the user wants in an effective small multiple display. We need a way to detect this.

Decision tree approach of guided EDA.

Trellis displays present a grid of plots of the same type showing the same variables conditioned on a Partitioner variable that determines the subsets of points shown in each plot. TCrucial to the layout of small multiples is the selection of a Partitioner variable to specify faceting into plots by rows and/or columns. This choice can then be repeated to facet the resulting plots further applying crossing and nesting.

Combinations of quality metrics or automatic selection of quality metric, more robust metrics.

Drill down (change level of aggregation) to create small multiples (should also be able to use permutation tests)



\section{Conclusion}

Small multiple displays are a powerful mechanism to analyze subsets of a visualized data relationship conditioned on another variable(s). Multidimensional datasets offer a challenge due to the combinatorics of the choice of variables for such a partitioning of the data. In this paper, we described a method for automatically ranking the small multiple displays created by the partitioning variables in a data set. We described a set of goodness criteria for small multiple displays that favors fewer partitions, have visually rich patterns that are well-supported by data observations and are different from the patterns seen in the unpartitioned view of the same data.

Here, we focus on scatterplots, as the primary data view, and scagnostics, as measures of visual patterns, to illustrate our method of evaluating small multiple displays. Our method can incorporate a wide range of existing quality measures for different visualization types, making our approach very general. Our use of a randomized permutation test allows our method to detect and discount non-informative or spurious patterns in small multiple displays.

The basis of our approach---the combination of cognostics and non-parametric tests---is very general and there is much more work to be done exploring this area. Such approaches will provide visualization users with rich tools that help them explore their data faster and more accurately.


%% if specified like this the section will be committed in review mode
\acknowledgments{
Acknowledgements blinded for review.}

\bibliographystyle{abbrv}
%%use following if all content of bibtex file should be shown
%\nocite{*}
\bibliography{paper2015}
\end{document}

